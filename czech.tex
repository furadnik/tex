% czech localisation
\usepackage[czech]{babel}
\renewcommand{\lstlistingname}{Ukázka kódu}
\renewcommand{\lstlistlistingname}{Seznam ukázek kódu}
% also: defi, thm, algo, glossary

\def\AlgIn{\emph{Vstup: }}
\def\AlgOut{\emph{Výstup: }}

\crefname{axiomsi}{axiom}{axiomy}
\Crefname{axiomsi}{Axiom}{Axiomy}
\crefname{invariantsi}{invariant}{invarianty}
\Crefname{invariantsi}{Invariant}{Invarianty}

\theoremstyle{Sdefi}
\newtheorem{defi}[thms]{Definice}
\newtheorem{notation}[thms]{Značení}

\theoremstyle{Sthm}
\newtheorem{thm}[thms]{Věta}
\newtheorem{lemma}[thms]{Lemma}
\newtheorem{algo}[thms]{Algoritmus}
\newtheorem{obs}[thms]{Pozorování}
\newtheorem{prop}[thms]{Tvrzení}
\newtheorem{invar}[thms]{Invariant}
\newtheorem{cor}[thms]{Důsledek}

\theoremstyle{Sex}
\newtheorem{examp}[thms]{Příklad}
\newtheorem{exercise}[thms]{Cvičení}
\newtheorem{fact}[thms]{Fakt}
\newtheorem{note}[thms]{Poznámka}


\newcommand{\glostitle}{Seznam zkratek}
